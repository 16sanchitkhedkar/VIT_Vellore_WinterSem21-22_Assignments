\documentclass[12pt]{article}
\usepackage[utf8]{inputenc}
\usepackage{graphicx} % Allows you to insert figures
\usepackage{amsmath} % Allows you to do equations
\usepackage{fancyhdr} % Formats the header
\usepackage{geometry} % Formats the paper size, orientation, and margins
\usepackage[style=authoryear-ibid,backend=biber]{biblatex} % Allows you to do citations - does Harvard style and compatible with Zotero
\addbibresource{Example.bib} % Tells LaTeX where the citations are coming from. This is imported from Zotero
\usepackage[english]{babel}
\usepackage{csquotes}
\usepackage{background}
\usepackage{minted}
\usepackage{amssymb}
\renewcommand*{\nameyeardelim}{\addcomma\space} % Adds comma in in-text citations
\linespread{1.5} % About 1.5 spacing in Word
\setlength{\parindent}{0pt} % No paragraph indents
\setlength{\parskip}{1em} % Paragraphs separated by one line
\renewcommand{\headrulewidth}{0pt} % Removes line in header
\geometry{a4paper, portrait, margin=1in}
\setlength{\headheight}{14.49998pt}
\backgroundsetup{scale=1,angle=0,opacity=0.175,contents={\includegraphics[scale=0.25]{1200px-Vellore_Institute_of_Technology_seal_2017.png}}}


\begin{document}
\begin{titlepage}
\NoBgThispage
   \begin{center}
        \begin{figure}[h] % h - Place the float here, i.e., approximately at the same point it occurs in the source text (however, not exactly at the spot)
        \centering
        \includegraphics[width=15cm]{1583124354phpJTtnK5.png}
        \end{figure}

        \Huge{Digital Assignment 1}

        \vspace{0.5cm}
        \LARGE{20BIT0406 - Sanchit Sandeep Khedkar}
       
        \vspace{2.5 cm}

        \vspace{0.25 cm}
        \Large{MAT3004 - Applied Linear Algebra}
        \large{VL2021220500257}
       

       \vfill
    \end{center}
\end{titlepage}
\newpage
Q1.The orthogonal projection of the vector (2, 0, −1, 3) on the plane spanned by (−1, 1, 0, 1) and (0, 1, 1, 1) in \( \mathbb{R}^4 \) is \( \frac{1}{5} \)(1, a, b, a) where a = \rule{0.5cm}{0.15mm} and b = \rule{0.5cm}{0.15mm}. The matrix which implements this orthogonal projection is \( \frac{1}{5} \)\begin{bmatrix}
c & −d & e & −d\\
−d & e & d & e\\
e & d & c & d\\
−d & e & d & e
\end{bmatrix} where c = \rule{0.5cm}{0.15mm}, d = \rule{0.5cm}{0.15mm}, and e = \rule{0.5cm}{0.15mm}.
\begin{figure}[h] % h - Place the float here, i.e., approximately at the same point it occurs in the source text (however, not exactly at the spot)
\centering
\includegraphics[scale=0.15]{2022-04-29 21-37-46 - ALA DA - p1.jpg}
\end{figure}
\begin{figure}[h] % h - Place the float here, i.e., approximately at the same point it occurs in the source text (however, not exactly at the spot)
\centering
\includegraphics[width=\textwidth]{2022-04-29 21-37-46 - ALA DA - p2.jpg}
\end{figure}
\newpage
\newpage
Q2. Let P be the orthogonal projection of \( \mathbb{R}^3 \) onto the subspace spanned by the vectors (1, 0, 1) and (1, 1, −1). Then [P]=\ \begin{bmatrix}
 a & b & c\\
 b & b & -b\\
 c & -b & a
 \end{bmatrix} a = \rule{0.5cm}{0.15mm}, b = \rule{0.5cm}{0.15mm}, c = \rule{0.5cm}{0.15mm}.
 \begin{figure}[h] % h - Place the float here, i.e., approximately at the same point it occurs in the source text (however, not exactly at the spot)
 \centering
 \includegraphics[scale=0.15]{2022-04-29 21-37-46 - ALA DA - p3.jpg}
 \end{figure}
\end{document}
