\documentclass[12pt]{article}
\usepackage[utf8]{inputenc}
\usepackage{graphicx} % Allows you to insert figures
\usepackage{amsmath} % Allows you to do equations
\usepackage{fancyhdr} % Formats the header
\usepackage{geometry} % Formats the paper size, orientation, and margins
\usepackage[style=authoryear-ibid,backend=biber]{biblatex} % Allows you to do citations - does Harvard style and compatible with Zotero
\addbibresource{Example.bib} % Tells LaTeX where the citations are coming from. This is imported from Zotero
\usepackage[english]{babel}
\usepackage{csquotes}
\usepackage{background}
\usepackage{minted}
\renewcommand*{\nameyeardelim}{\addcomma\space} % Adds comma in in-text citations
\linespread{1.5} % About 1.5 spacing in Word
\setlength{\parindent}{0pt} % No paragraph indents
\setlength{\parskip}{1em} % Paragraphs separated by one line
\renewcommand{\headrulewidth}{0pt} % Removes line in header
\geometry{a4paper, portrait, margin=1in}
\setlength{\headheight}{14.49998pt}
\backgroundsetup{scale=1,angle=0,opacity=0.175,contents={\includegraphics[scale=0.25]{1200px-Vellore_Institute_of_Technology_seal_2017.png}}}


\begin{document}
\begin{titlepage}
\NoBgThispage
   \begin{center}
        \begin{figure}[h] % h - Place the float here, i.e., approximately at the same point it occurs in the source text (however, not exactly at the spot)
        \centering
        \includegraphics[width=15cm]{1583124354phpJTtnK5.png}
        \end{figure}

        \Huge{Digital Assignment 1}

        \vspace{0.5cm}
        \LARGE{20BIT0406 - Sanchit Sandeep Khedkar}
       
        \vspace{2.5 cm}

        \vspace{0.25 cm}
        \Large{ITE2002 - Operating Systems}
        \large{VL2021220500440}
       

       \vfill
    \end{center}
\end{titlepage}
\newpage
\section*{Windows Operating System}
\subsection*{History}
In 1988, Microsoft decided to develop a “new technology” (NT) portable operating system that supported both the OS/2 and POSIX APIs
Originally, NT was supposed to use the OS/2 API as its native environment but during development NT was changed to use the Win32 API, reflecting the popularity of Windows 3.0. Windows NT Version 4.0 adopted the Windows 95 user interface and incorporated Internet web-server andweb-browser software. In addition, user-interface routines and
all graphics code were moved into the kernel to improve performance (with
the side effect of decreased system reliability and significant loss of security).
Although previous versions of NT had been ported to other microprocessor
architectures (including a brief 64-bit port to Alpha AXP 64), the Windows
2000 version, released in February 2000, supported only IA-32-compatible processors
due to marketplace factors. Windows 2000 incorporated significant
changes. It added Active Directory (an X.500-based directory service), better
networking and laptop support, support for plug-and-play devices, a distributed
file system, and support for more processors and more memory.
\subsubsection*{Windows XP, Vista, 7}
In October 2001, Windows XP was released as both an update to the Windows
2000 desktop operating system and a replacement for Windows 95/98. In April
2003, the server edition of Windows XP (called Windows Server 2003) became
available.Windows XP updated the graphical user interface (GUI) with a visual
design that took advantage of more recent hardware advances and many
new ease-of-use features. Numerous features were added to automatically
repair problems in applications and the operating system itself. Because of
these changes,Windows XP provided better networking and device experience
(including zero-configuration wireless, instant messaging, streaming media,
and digital photography/video). Windows Server 2003 provided dramatic
performance improvements for large multiprocessors systems, as well as better
reliability and security than earlier Windows operating systems.
The long-awaited update to Windows XP, called Windows Vista, was
released in January 2007, but it was not well received. Although Windows
Vista included many improvements that later continued into Windows 7, these
improvements were overshadowed by Windows Vista’s perceived sluggishness
and compatibility problems. Microsoft responded to criticisms of Windows
Vista by improving its engineering processes and working more closely
with the makers of Windows hardware and applications.
The result was Windows 7, which was released in October 2009, along
with corresponding server edition called Windows Server 2008 R2. Among
the significant engineering changes was the increased use of event tracing
rather than counters or profiling to analyze system behavior. Tracing runs
constantly in the system, watching hundreds of scenarios execute. Scenarios
include process startup and exit, file copy, and web-page load, for example.
When one of these scenarios fails, or when it succeeds but does not perform
well, the traces can be analyzed to determine the cause.
\subsubsection*{Windows 8}
Three years later, in October 2012—amid an industry-wide pivot toward
mobile computing and the world of apps—Microsoft released Windows 8,
which represented the most significant change to the operating system since
Windows XP. Windows 8 included a new user interface (named Metro) and a
new programming model API (named WinRT). It also included a new way of
managing applications (which ran under a new sandbox mechanism) through
a package system that exclusively supported the new Windows Store, a competitor
to the Apple App Store and the Android Store.
\subsubsection*{Windows 10}
With the release of Windows 10 in July 2015 and its server companion,
Windows Server 2016, in October 2016, Microsoft shifted to a “Windows-as-a-Service” (WaaS) model (with included periodic functionality improvements).
Windows 10 receives monthly incremental improvements called “feature
rollups,” as well as eight-month feature releases called “updates.” Windows 10 reintroduced the start menu, restored keyboard support, and
deemphasized full-screen applications and live tiles. From the user’s perspective,
these changes brought back the ease of use that users expected from
Windows-based desktop operating systems. Additionally, Metro (which was
renamed Modern) was redesigned so that Windows Store–packaged applications
could be run on the regular desktop side by side with legacy applications.
Finally, a new mechanism called the Windows Desktop Bridge made it possible
to place Win32 applications in the Windows Store, mitigating the lack of
applications written specifically for the newer systems. Meanwhile, Microsoft
added support for C++11, C++14, and C++17 in the Visual Studio product,
and many new APIs were added to the traditional Win32 programming API.
\subsection*{Design Principles}
\subsubsection*{Security}
Windows traditionally based security on discretionary access controls. System
objects, including files, registry keys, and kernel synchronization objects,
are protected by access-control lists (ACLs). ACLs are vulnerable
to user and programmer errors, however, as well as to the most common
attacks on consumer systems, in which the user is tricked into running
code, often while browsing the Web. Windows Vista introduced a mechanism
called integrity levels that acts as a rudimentary capability system for controlling
access. Objects and processes are marked as having no, low, medium,
or high system integrity. The integrity level determines what rights the objects
and processes will have.Windows 10 further strengthened the security model by introducing a
combination of attribute-based access control (ABAC) and claim-based access
control (CABC). Both features are used to implement dynamic access control
(DAC) on server editions, aswell as to support the capability-based system used
by Windows Store applications and by Modern and packaged applications.
Windows uses encryption as part of common protocols such as those used
to communicate securely with websites. Encryption is also used to protect user
files stored on secondary storage.Windows 7 and later versions allow users to easily encrypt entire volumes, aswell as removable storage devices such as USB
flash drives, with a feature called BitLocker. These types of security features focus on user and data security, but they
are vulnerable to highly privileged programs that parse arbitrary content
and that can be tricked due to programming errors into executing malicious
code. Therefore, Windows also includes security measures often referred to
as “exploit mitigations.” These measures include wide-scope mitigations such
as address-space layout randomization (ASLR), Data Execution Prevention
(DEP), Control-Flow Guard (CFG), and Arbitrary Code Guard (ACG), as well as
narrow-scope (targeted) mitigations specific to various exploitation techniques.
Because DEP prevents attacker-controlled data from being executed as
code, malicious developers moved on to code reuse attacks, in which existing
executable code inside the program is reused in unexpected ways. (Only
certain parts of the code are executed, and the flow is redirected from one
instruction stream to another.) ASLR thwarts many forms of such attacks by
randomizing the location of executable (and data) regions of memory, making
it harder for code-reuse attacks to know where existing code is located. This
safeguard makes it likely that a system under attack by a remote attacker will
fail or crash.Another important aspect of security is integrity. Windows offers several
digital signature facilities as part of its code integrity features. Windows uses
digital signatures to sign operating system binaries so that it can verify that
the files were produced by Microsoft or another known company. In non-IA-32
versions of Windows, the code integrity module is activated at boot to ensure
that all the loaded modules in the kernel have valid signatures, assuring that
they have not been tampered with. Additionally, ARM versions of Windows 8
extend the code integrity module with user-mode code integrity checks, which
validate that all user programs have been signed by Microsoft or delivered
through the Windows Store. Finally, enterprise versions of Windows 10 make it possible to opt in to a
new security feature called Device Guard. This mechanism allows organizations
to customize the digital signing requirements of their computer systems,
as well as blacklist and whitelist individual signing certificates or even binary
hashes. For example, an organization could choose to allow only user-mode
programs signed by Microsoft, Google, or Adobe to launch on their enterprise
computers.
\subsubsection*{Reliability}
Windows has extended
the tools for achieving reliability to include automatic analysis of source code
for errors, tests to detect validation failures, and an application version of the driver verifier that applies dynamic checking for many common user-mode
programming errors. Other improvements in reliability have resulted from
moving more code out of the kernel and into user-mode services. Windows
provides extensive support for writing drivers in user mode. System facilities
that were once in the kernel and are now in user mode include the renderer for
third-party fonts and much of the software stack for audio.
One of the most significant improvements in theWindows experience came
from adding memory diagnostics as an option at boot time. This addition is
especially valuable because so few consumer PCs have error-correcting memory.
Bad RAM that lacks error correction and detection can change the data
it stores—a change undetected by the hardware. The result is frustratingly
erratic behavior in the system. The availability of memory diagnostics canwarn
users of a RAM problem.Windows 10 took this even further by introducing runtime
memory diagnostics. If a machine encounters a kernel-mode crash more
than five times in a row, and the crashes cannot be pinpointed to a specific cause
or component, the kernelwill use idle periods to move memory contents, flush
system caches, and write repeated memory-testing patterns in all memory—
all to preemptively discover if RAM is damaged. Users can then be informed of
any issues without the need to reboot into the memory diagnostics tool at boot
time. Windows 7 also introduced a fault-tolerant memory heap. The heap learns
from application crashes and automatically adjusts memory operations carried
out by an application that has crashed. This makes the application more reliable
even if it contains common bugs such as using memory after freeing it or
accessing past the end of the allocation. Because such bugs can be exploited
by attackers,Windows 7 also includes amitigation for developers to block this
feature and immediately crash any application with heap corruption. This is a
very practical representation of the dichotomy that exists between the needs of
security and the needs of user experience.
\subsubsection*{Performance}
Windows NT was designed for symmetrical multiprocessing (SMP); on a
multiprocessor computer, several threads can run at the same time, even in the
kernel. On each CPU, Windows NT uses priority-based preemptive scheduling
of threads. Except while executing in the dispatcher or at interrupt level,
threads in any process running in Windows can be preempted by higherpriority
threads. Thus, the system responds quickly.
Windows XP further improved performance by reducing the code-path
length in critical functions and implementing more scalable locking protocols,
such as queued spinlocks and pushlocks. (Pushlocks are like optimized spinlockswith
read–write lock features.) The new locking protocols helped reduce
system bus cycles and included lock-free lists and queues, atomic read–modify
–write operations (like interlocked increment), and other advanced synchronization
techniques. By the timeWindows 7was developed, several major changes had come to
computing. The number of CPUs and the amount of physicalmemory available
in the largest multiprocessors had increased substantially, so quite a lot of effort
was put into further improving operating-system scalability.
The implementation of multiprocessing support in Windows NT used bitmasks
to represent collections of processors and to identify, for example,which
set of processors a particular thread could be scheduled on. These bitmasks
were defined as fitting within a singleword of memory, limiting the number of
processors supported within a system to 64 on a 64-bit system and 32 on a 32-bit
system. Thus, Windows 7 added the concept of processor groups to represent
a collection of up to 64 processors. Multiple processor groups could be created,
accommodating a total of more than 64 processors. To support task-based parallelism, the AMD64 ports ofWindows 7 and later
versions provide a new form of user-mode scheduling (UMS). UMS allows
programs to be decomposed into tasks, and the tasks are then scheduled on
the available CPUs by a scheduler that operates in usermode rather than in the
kernel.
\subsubsection*{Extensibility}
Extensibility refers to the capability of an operating system to keep up with
advances in computing technology. To facilitate change over time, the developers
implemented Windows using a layered architecture. Even in the kernel,Windows uses a layered architecture, with loadable
drivers in the I/O system, so new file systems, new kinds of I/O devices, and
new kinds of networking can be added while the system is running. Windows also uses a client–server model like the Mach operating system
and supports distributed processing through remote procedure calls (RPCs) as
defined by the Open Software Foundation. These RPCs take advantage of an
executive component, called the advanced local procedure call (ALPC), that
implements highly scalable communication between separate processes on a
local machine.
\subsubsection*{Portability}
An operating system is portable if it can be moved from one CPU architecture
to another with relatively few changes.Windows was designed to be portable.
Like the UNIX operating system, Windows is written primarily in C and C++. Porting Windows to a new architecture mostly affects the Windows
kernel, since the user-mode code in Windows is almost exclusively written
to be architecture independent. To port Windows, the kernel’s architecturespecific
code must be rewritten for the target CPU, and sometimes conditional
compilation is needed in other parts of the kernel because of changes in major
data structures, such as the page-table format. The entire Windows system
must then be recompiled for the new CPU instruction set. Operating systems are sensitive not only to CPU architecture but also to CPU
support chips and hardware boot programs. The CPU and support chips are
collectively known as the chipset. These chipsets and the associated boot code
determine how interrupts are delivered, describe the physical characteristics of
each system, and provide interfaces to deeper aspects of the CPU architecture,
such as error recovery and power management. It would be burdensome to
have to port Windows to each type of support chip as well as to each CPU
architecture. Instead,Windows isolates most of the chipset-dependent code in
a dynamic link library (DLL), called the hardware-abstraction layer (HAL), that
is loaded with the kernel. Over the years, Windows has been ported to a number of different CPU
architectures: Intel IA-32-compatible 32-bit CPUs, AMD64-compatible and IA64
64-bit CPUs, and DEC Alpha, DEC Alpha AXP64, MIPS, and PowerPC CPUs.
Most of these CPU architectures failed in the consumer desktop market.When
Windows 7 shipped, only the IA-32 and AMD64 architectures were supported
on client computers, along with AMD64 on servers. With Windows 8, 32-bit
ARM was added, and Windows 10 now supports ARM64 as well.
\end{document}
